% lecture01.tex — Lecture 1: Introduction & History of Planetary Science
% Introduction to Planetary Science, Kapteyn Institute

\documentclass[aspectratio=169, 11pt]{beamer}

% ── Theme and macros ────────────────────────────────────────
\usepackage{../common/beamerthemeIPS}
% macros.tex — Shared math macros for IPS lecture slides
% Mirrors definitions in book/_config.yml for consistency

% ── Derivatives ─────────────────────────────────────────────
\newcommand{\dv}[2]{\frac{\mathrm{d} #1}{\mathrm{d} #2}}
\newcommand{\dd}{\mathrm{d}}
\newcommand{\pdv}[2]{\frac{\partial #1}{\partial #2}}

% ── Solar quantities ────────────────────────────────────────
\newcommand{\Msun}{M_{\odot}}
\newcommand{\Rsun}{R_{\odot}}
\newcommand{\Lsun}{L_{\odot}}

% ── Planetary quantities ────────────────────────────────────
\newcommand{\Mearth}{M_{\oplus}}
\newcommand{\Rearth}{R_{\oplus}}
\newcommand{\Mjup}{M_{\mathrm{J}}}
\newcommand{\Rjup}{R_{\mathrm{J}}}

% ── Constants ───────────────────────────────────────────────
\newcommand{\kB}{k_{\mathrm{B}}}

\graphicspath{{figures/}{../common/}}

% ── Metadata ────────────────────────────────────────────────
\title{Lecture 1: Introduction \& History\\of Planetary Science}
\subtitle{Introduction to Planetary Science --- WBAS002-05}
\author[L1]{Tim Lichtenberg}
\institute{Kapteyn Astronomical Institute\\University of Groningen}
\date{September 2026}

% ════════════════════════════════════════════════════════════
\begin{document}

% ── Title slide ─────────────────────────────────────────────
\begin{frame}[plain,noframenumbering]
  \titlepage
\end{frame}

% ── Learning objectives ─────────────────────────────────────
\begin{frame}{Learning Objectives}
  By the end of this lecture, you will be able to:
  \vskip8pt
  \begin{itemize}
    \item Describe the scope of planetary science and its three driving questions
    \item Explain how the field evolved from antiquity to the space age
    \item Identify the key properties and classification of solar system bodies
    \item Derive the stellar mass from planetary orbital parameters
  \end{itemize}
\end{frame}

% ── Course overview ─────────────────────────────────────────
\begin{frame}{Course Overview}
  \begin{columns}[T]
    \column{0.55\textwidth}
    \textbf{14 lectures} spanning:
    \begin{itemize}
      \item Formation \& orbital dynamics
      \item Heat, interiors, differentiation
      \item Atmospheres \& surfaces
      \item Terrestrial planets in detail
      \item Gas \& ice giants, small bodies
      \item Exoplanets \& astrobiology
    \end{itemize}
    \vskip8pt
    \textbf{Assessment:} Mid-term (30\%) + Final (70\%)

    \column{0.42\textwidth}
    \begin{block}{Three driving questions}
      \begin{enumerate}
        \item How did our solar system form?
        \item What makes a planet habitable?
        \item Are we alone?
      \end{enumerate}
    \end{block}
  \end{columns}
\end{frame}


% ════════════════════════════════════════════════════════════
\section{A Pale Blue Dot}
% ════════════════════════════════════════════════════════════

% ── Pale Blue Dot image ─────────────────────────────────────
\begin{frame}{A Pale Blue Dot}
  \begin{columns}[c]
    \column{0.45\textwidth}
    \centering
    \includegraphics[height=0.75\textheight]{pale_blue_dot}
    \source{NASA/JPL-Caltech (1990)}

    \column{0.52\textwidth}
    14 February 1990 --- Voyager~1 turned its camera back toward the inner solar system from $\sim$6 billion~km ($\sim$40~AU).
    \vskip10pt
    \textit{``Look again at that dot. That's here.\\That's home. That's us.''}
    \vskip4pt
    \hfill--- Carl Sagan (1994)
    \vskip10pt
    Earth appears as a tiny speck, less than a single pixel, suspended in a scattered beam of sunlight.
  \end{columns}
\end{frame}

% ── Driving question 1 ─────────────────────────────────────
\begin{frame}{Question 1: How Did Our Solar System Form?}
  \begin{itemize}
    \item Before 1992: zero confirmed exoplanets
    \item Today: $>5{,}700$ confirmed exoplanets in $>4{,}300$ systems
    \item $\sim$0.4--0.6 rocky, habitable-zone planets per Sun-like star (Bryson et al.\ 2021)
    \item[\color{ipsAccent}$\Rightarrow$] \textbf{Hundreds of millions} of such worlds in the Milky Way
    \item Yet we have detailed knowledge of only \textbf{one} planetary system
  \end{itemize}
  \vskip12pt
  \begin{block}{}
    Is our solar system typical --- or unusual?
  \end{block}
\end{frame}

% ── Driving question 2 ─────────────────────────────────────
\begin{frame}{Question 2: What Makes a Planet Habitable?}
  \begin{itemize}
    \item Venus, Earth, and Mars formed from the same protoplanetary disk
    \item All three may have had early liquid water
    \item Today:
    \begin{itemize}
      \item \textbf{Venus:} 460\,\textdegree C surface, 90~bar $\mathrm{CO_2}$
      \item \textbf{Earth:} Oceans, biosphere, plate tectonics
      \item \textbf{Mars:} Cold desert, atmosphere $<1\%$ of Earth's
    \end{itemize}
  \end{itemize}
  \vskip8pt
  \begin{alertblock}{}
    Why did these three siblings diverge so dramatically?
  \end{alertblock}
\end{frame}

% ── Driving question 3 ─────────────────────────────────────
\begin{frame}{Question 3: Are We Alone?}
  \begin{itemize}
    \item JWST is characterising atmospheres on rocky exoplanets
    \item TRAPPIST-1 system: seven Earth-sized planets, three in the habitable zone
    \item Within your careers, this question may become answerable
    \item But only if we understand what makes a planet habitable in the first place
  \end{itemize}
  \vskip12pt
  \keyresult{Every topic in this course connects back to these three questions.}
\end{frame}


% ════════════════════════════════════════════════════════════
\section{What Is a Planet?}
% ════════════════════════════════════════════════════════════

% ── Etymology and history ───────────────────────────────────
\begin{frame}{What Is a Planet?}
  \begin{itemize}
    \item Greek: \textit{planetes} ($\pi\lambda\alpha\nu\acute{\eta}\tau\eta\varsigma$) = ``wanderer''
    \item Five visible wanderers: Mercury, Venus, Mars, Jupiter, Saturn
    \item With the Sun and Moon $\rightarrow$ seven days of the week
    \item Planet count unchanged for over two millennia
  \end{itemize}
  \vskip10pt
  \textbf{Telescopic discoveries:}
  \begin{itemize}
    \item 1781 --- Herschel discovers \textbf{Uranus}
    \item 1846 --- Galle observes \textbf{Neptune} (predicted by Le Verrier)
    \item 1930 --- Tombaugh discovers \textbf{Pluto}
  \end{itemize}
\end{frame}

% ── Pluto controversy ───────────────────────────────────────
\begin{frame}{The Pluto Problem}
  \begin{itemize}
    \item Pluto was always an oddity: small, icy, eccentric \& inclined orbit
    \item 1990s--2000s: discovery of many large Kuiper Belt objects
    \item 2005: discovery of \textbf{Eris} --- comparable in size to Pluto
    \item The question became unavoidable:
  \end{itemize}
  \vskip8pt
  \begin{alertblock}{}
    Either these new objects are also planets, or Pluto is not.
  \end{alertblock}
\end{frame}

% ── IAU definition ──────────────────────────────────────────
\begin{frame}{IAU Definition (2006)}
  \textbf{Resolution B5} --- A \textbf{planet} in the solar system must:
  \vskip6pt
  \begin{enumerate}
    \item Be in orbit around the Sun
    \item Have sufficient mass for self-gravity to achieve hydrostatic equilibrium (nearly round)
    \item Have \textbf{cleared the neighbourhood} around its orbit
  \end{enumerate}
  \vskip10pt
  \begin{columns}[T]
    \column{0.48\textwidth}
    \textbf{Result:}
    \begin{itemize}
      \item 8 planets
      \item 5 recognised dwarf planets:\\Ceres, Pluto, Eris, Makemake, Haumea
    \end{itemize}

    \column{0.48\textwidth}
    \textbf{Criticism:}
    \begin{itemize}
      \item ``Clearing'' is not precisely defined
      \item Depends on heliocentric distance
      \item Earth wouldn't clear its zone at Neptune's orbit
    \end{itemize}
  \end{columns}
\end{frame}

% ── Geophysical definition ──────────────────────────────────
\begin{frame}{Alternative: Geophysical Definition}
  \begin{block}{Runyon et al.\ (2017)}
    Any body massive enough to achieve \textbf{hydrostatic equilibrium} is a planet, regardless of orbital dynamics.
  \end{block}
  \vskip8pt
  \begin{itemize}
    \item Under this definition: $>$100 planets in the solar system
    \item Includes large moons (Titan, Europa, Ganymede, \ldots)
  \end{itemize}
  \vskip12pt
  \keyresult{For this course, the exact classification matters less than the physics. The solar system contains a continuous spectrum of objects --- from dust grains to gas giants --- governed by the same physical laws.}
\end{frame}


% ════════════════════════════════════════════════════════════
\section{Brief History of Planetary Science}
% ════════════════════════════════════════════════════════════

% ── Ancient astronomy ───────────────────────────────────────
\begin{frame}{Ancient \& Pre-Telescopic Astronomy}
  \begin{itemize}
    \item \textbf{$\sim$1800 BCE}: Babylonian astronomers systematically track planetary positions
    \item \textbf{$\sim$350 BCE}: Aristotle's geocentric spheres
    \item \textbf{$\sim$150 CE}: Ptolemy's epicyclic system --- standard for $>$1000 years
    \item \textbf{1543}: Copernicus publishes \textit{De revolutionibus}\\
          $\rightarrow$ Sun at the centre
    \item \textbf{1609--1619}: Kepler's three laws of planetary motion\\
          $\rightarrow$ Circles replaced by ellipses
    \item \textbf{1687}: Newton's \textit{Principia}\\
          $\rightarrow$ All three laws from \textbf{universal gravitation}
  \end{itemize}
\end{frame}

% ── Copernican system figure ────────────────────────────────
\begin{frame}{The Copernican Revolution}
  \begin{columns}[c]
    \column{0.45\textwidth}
    \centering
    \includegraphics[width=\textwidth]{copernican_system}
    \source{Copernicus, \textit{De revolutionibus} (1543)}

    \column{0.50\textwidth}
    \begin{itemize}
      \item Heliocentric model: Sun (\textit{Sol}) at the centre
      \item Earth (\textit{Terra}) orbits like any other planet
      \item Conceptual shift from 1400~years of geocentric cosmology
      \item Kepler refined circular orbits $\rightarrow$ \textbf{ellipses}
      \item Newton showed: single law of gravity explains all three of Kepler's laws
    \end{itemize}
  \end{columns}
\end{frame}

% ── Galileo ─────────────────────────────────────────────────
\begin{frame}{Galileo's Telescope (1610)}
  \textbf{Key observations that transformed planetary science:}
  \vskip6pt
  \begin{itemize}
    \item Four moons orbiting Jupiter (Galilean moons)\\
          $\rightarrow$ Not everything orbits Earth
    \item Phases of Venus\\
          $\rightarrow$ Venus orbits the Sun
    \item Saturn's rings (could not interpret their structure)
    \item Mountains and craters on the Moon\\
          $\rightarrow$ Celestial bodies are \textit{physical worlds}
  \end{itemize}
  \vskip6pt
  \keyresult{Planetary science shifts from pure mathematics to a \textbf{physical science}.}
\end{frame}

% ── Sidereus Nuncius ────────────────────────────────────────
\begin{frame}{Sidereus Nuncius (1610)}
  \begin{columns}[c]
    \column{0.35\textwidth}
    \centering
    \includegraphics[height=0.72\textheight]{sidereus_nuncius}
    \source{Galileo Galilei, public domain}

    \column{0.60\textwidth}
    \begin{itemize}
      \item \textit{Sidereus Nuncius} = ``Starry Messenger''
      \item Sketches of Jupiter and its four largest moons\\(\textit{Medicea Sidera})
      \item Recorded changing positions over successive nights
      \item Direct evidence: not all celestial bodies orbit Earth
      \item One of the most important publications in the history of science
    \end{itemize}
  \end{columns}
\end{frame}

% ── 19th century ────────────────────────────────────────────
\begin{frame}{The 19th Century: Spectroscopy \& Discovery}
  \begin{itemize}
    \item Improving telescopes reveal surface features on Mars, Jupiter's Great Red Spot, Saturn's ring structure
    \item \textbf{Spectroscopy} enables remote atmospheric composition measurements for the first time
    \item Photography enables systematic surveys
    \item The known solar system steadily expands:\\
    Uranus (1781) $\rightarrow$ Neptune (1846) $\rightarrow$ Pluto (1930)
  \end{itemize}
\end{frame}

% ── Space age begins ────────────────────────────────────────
\begin{frame}{The Space Age Begins}
  \textbf{14 December 1962}: Mariner~2 flies past Venus\\
  $\rightarrow$ First successful planetary flyby
  \vskip8pt
  \begin{itemize}
    \item Revealed Venus's extreme surface temperature ($\sim$460\,\textdegree C)
    \item Overturned speculation about habitable conditions
    \item \textbf{1965}: Mariner~4 returns first close-up images of Mars\\
          $\rightarrow$ Cratered, apparently dead world (not canals!)
    \item \textbf{1970}: Venera~7 --- first landing on another planet (Venus)
    \item \textbf{1976}: Viking landers --- first life-detection experiments on Mars
  \end{itemize}
\end{frame}

% ── Voyager grand tour ──────────────────────────────────────
\begin{frame}{Voyager Grand Tour}
  \begin{columns}[c]
    \column{0.52\textwidth}
    \centering
    \includegraphics[width=\textwidth]{voyager2_trajectory}
    \source{NASA/JPL, public domain}

    \column{0.45\textwidth}
    \begin{itemize}
      \item Launched 1977
      \item Exploited a rare planetary alignment ($\sim$once every 175 years)
      \item Gravity assists at each planet
      \item Jupiter (1979), Saturn (1981), Uranus (1986), Neptune (1989)
      \item Voyager~1 now in interstellar space
    \end{itemize}
  \end{columns}
\end{frame}

% ── Exoplanet revolution ───────────────────────────────────
\begin{frame}{The Exoplanet Revolution}
  \begin{itemize}
    \item \textbf{1992}: Wolszczan \& Frail --- planets orbiting a \textbf{pulsar}\\
          $\rightarrow$ First confirmed exoplanets
    \item \textbf{1995}: Mayor \& Queloz --- \textbf{51 Pegasi b}\\
          Hot Jupiter with 4.2-day orbit $\rightarrow$ challenged all formation theories
    \item \textbf{2009--2018}: Kepler mission discovers thousands of transiting exoplanets
    \item \textbf{2018--}: TESS surveys the brightest nearby stars
    \item \textbf{2022--}: JWST characterises exoplanet atmospheres
  \end{itemize}
\end{frame}

% ── Transit method ──────────────────────────────────────────
\begin{frame}{The Transit Method}
  \begin{columns}[c]
    \column{0.50\textwidth}
    \centering
    \includegraphics[width=0.95\textwidth]{transit_light_curve}
    \source{CielProfond, CC BY-SA 4.0}

    \column{0.47\textwidth}
    \begin{itemize}
      \item Planet passes in front of host star
      \item Blocks a fraction of starlight:
      $$\frac{\Delta F}{F} = \left(\frac{R_p}{R_\star}\right)^2$$
      \item Gives the planet's \textbf{radius}
      \item Used by Kepler, TESS, JWST
      \item Has discovered the majority of known exoplanets
    \end{itemize}
  \end{columns}
\end{frame}

% ── Modern planetary science ────────────────────────────────
\begin{frame}{Planetary Science Today}
  \begin{block}{}
    Today, planetary science integrates:
  \end{block}
  \vskip4pt
  \begin{columns}[T]
    \column{0.48\textwidth}
    \begin{itemize}
      \item Astronomy
      \item Physics
      \item Chemistry
    \end{itemize}
    \column{0.48\textwidth}
    \begin{itemize}
      \item Geology
      \item Atmospheric science
      \item Biology
    \end{itemize}
  \end{columns}
  \vskip12pt
  It spans scales from \textbf{dust grains} in protoplanetary disks to the \textbf{demographics of planetary systems} across the Galaxy.
\end{frame}


% ════════════════════════════════════════════════════════════
\section{Overview of the Solar System}
% ════════════════════════════════════════════════════════════

% ── Architecture and scale ──────────────────────────────────
\begin{frame}{Solar System Architecture}
  From the Sun ($\Rsun = 6.96 \times 10^8$~m) to the Oort Cloud ($\sim 10^4$--$10^5$~AU):
  \vskip6pt
  \begin{itemize}
    \item \textbf{Inner solar system} (0.4--1.5~AU):\\
          Mercury, Venus, Earth, Mars --- small, dense, rocky
    \item \textbf{Asteroid belt} (2.1--3.3~AU):\\
          Total mass $\sim 4 \times 10^{-4}\;\Mearth$ --- dominated by Ceres
    \item \textbf{Outer solar system} (5.2--30.1~AU):\\
          Jupiter, Saturn (gas giants); Uranus, Neptune (ice giants)
    \item \textbf{Kuiper Belt} (30--50~AU):\\
          Icy bodies: Pluto, Eris, Makemake, \ldots
    \item \textbf{Oort Cloud} ($10^4$--$10^5$~AU):\\
          Source of long-period comets (not directly observed)
  \end{itemize}
\end{frame}

% ── Planet sizes figure ─────────────────────────────────────
\begin{frame}{Relative Sizes of the Planets}
  \centering
  \includegraphics[width=0.82\textwidth]{planet_sizes}
  \source{Lsmpascal, CC BY-SA 3.0}
  \vskip4pt
  {\small Top: Jupiter, Saturn, Uranus, Neptune \quad|\quad Bottom: Earth, Venus, Mars, Mercury\\
  Jupiter's diameter is $\sim$29$\times$ that of Mercury.}
\end{frame}

% ── Properties table (inner planets) ────────────────────────
\begin{frame}{Planetary Properties: Inner Solar System}
  \centering
  \small
  \begin{tabular}{l r r r r r r}
    \toprule
    \textbf{Planet} & \textbf{Mass} ($\Mearth$) & \textbf{Radius} ($\Rearth$) & $\boldsymbol{a}$ (AU) & $\boldsymbol{P}$ (yr) & $\boldsymbol{e}$ & $\boldsymbol{\rho}$ (kg\,m$^{-3}$) \\
    \midrule
    Mercury & 0.055 & 0.383 & 0.387 & 0.241 & 0.206 & 5427 \\
    Venus   & 0.815 & 0.949 & 0.723 & 0.615 & 0.007 & 5243 \\
    Earth   & 1.000 & 1.000 & 1.000 & 1.000 & 0.017 & 5514 \\
    Mars    & 0.107 & 0.532 & 1.524 & 1.881 & 0.093 & 3934 \\
    \bottomrule
  \end{tabular}
  \vskip8pt
  \source{NASA Planetary Fact Sheet}
  \vskip6pt
  All have $\rho > 3900$~kg\,m$^{-3}$ --- \textbf{rock and metal} dominate.
\end{frame}

% ── Properties table (outer planets) ────────────────────────
\begin{frame}{Planetary Properties: Outer Solar System}
  \centering
  \small
  \begin{tabular}{l r r r r r r}
    \toprule
    \textbf{Planet} & \textbf{Mass} ($\Mearth$) & \textbf{Radius} ($\Rearth$) & $\boldsymbol{a}$ (AU) & $\boldsymbol{P}$ (yr) & $\boldsymbol{e}$ & $\boldsymbol{\rho}$ (kg\,m$^{-3}$) \\
    \midrule
    Jupiter & 317.8  & 11.21 & 5.203  & 11.86  & 0.049 & 1326 \\
    Saturn  & 95.16  & 9.45  & 9.537  & 29.46  & 0.054 & 687  \\
    Uranus  & 14.54  & 4.01  & 19.19  & 84.01  & 0.047 & 1271 \\
    Neptune & 17.15  & 3.88  & 30.07  & 164.8  & 0.009 & 1638 \\
    \bottomrule
  \end{tabular}
  \vskip8pt
  \source{NASA Planetary Fact Sheet}
  \vskip6pt
  All have $\rho < 1700$~kg\,m$^{-3}$ --- \textbf{gas and ice} dominate.\\
  Saturn is famously less dense than water ($\rho_\mathrm{water} = 1000$~kg\,m$^{-3}$).
\end{frame}

% ── Two key patterns ────────────────────────────────────────
\begin{frame}{Two Key Patterns}
  \begin{enumerate}
    \item \textbf{Density decreases with distance}\\[4pt]
          Inner: $\rho > 3900$~kg\,m$^{-3}$ (rock \& metal)\\
          Outer: $\rho < 1700$~kg\,m$^{-3}$ (gas \& ice)\\[4pt]
          $\Rightarrow$ Reflects the temperature structure of the protoplanetary disk

    \vskip10pt
    \item \textbf{Mass is concentrated in Jupiter}\\[4pt]
          Jupiter $>$ twice the mass of all other planets combined\\[4pt]
          $\Rightarrow$ We will quantify this in the blackboard derivation
  \end{enumerate}
\end{frame}

% ── Classification ──────────────────────────────────────────
\begin{frame}{Classification of Planets}
  \begin{description}
    \item[Terrestrial] Mercury, Venus, Earth, Mars\\
    Rocky surfaces, iron cores, thin or no atmospheres\\
    (Venus: massive $\mathrm{CO_2}$ atmosphere is the exception)

    \vskip6pt
    \item[Gas giants] Jupiter, Saturn\\
    Massive $\mathrm{H}$/$\mathrm{He}$ envelopes, no well-defined solid surface\\
    Likely rocky/icy cores at high pressure

    \vskip6pt
    \item[Ice giants] Uranus, Neptune\\
    Interiors dominated by $\mathrm{H_2O}$, $\mathrm{NH_3}$, $\mathrm{CH_4}$ under extreme pressures\\
    Topped by $\mathrm{H}$/$\mathrm{He}$ atmospheres
  \end{description}
\end{frame}

% ── Terrestrial planets figure ──────────────────────────────
\begin{frame}{The Terrestrial Planets}
  \centering
  \includegraphics[width=0.72\textwidth]{terrestrial_planets}
  \source{NASA/JPL, public domain}
  \vskip4pt
  {\small Mercury, Venus, Earth, Mars at approximate relative scale.\\
  Factor of $\sim$18 in mass, dramatically different evolutionary paths.}
\end{frame}


% ════════════════════════════════════════════════════════════
\section{Blackboard Derivation: Solar Mass}
% ════════════════════════════════════════════════════════════

% ── Setup ───────────────────────────────────────────────────
\begin{frame}{Deriving the Solar Mass from Kepler's Third Law}
  \textbf{Goal:} Estimate $\Msun$ from Earth's orbital parameters.
  \vskip10pt
  \textbf{Setup:} Planet of mass $M_p$ in a circular orbit of radius $r$ around a star of mass $M_*$.
  \vskip8pt
  Gravitational force provides centripetal acceleration:
  $$\frac{G\, M_*\, M_p}{r^2} = \frac{M_p\, v^2}{r}$$
  \vskip6pt
  where $v = 2\pi r / P$ is the orbital velocity.
\end{frame}

% ── Derivation step 1 ──────────────────────────────────────
\begin{frame}{Step 1: Substitute Orbital Velocity}
  Substitute $v = 2\pi r / P$ into the force balance:
  $$\frac{G\, M_*\, M_p}{r^2} = \frac{M_p}{r} \cdot \frac{4\pi^2 r^2}{P^2}$$
  \vskip10pt
  Cancel $M_p$ and simplify:
  $$\frac{G\, M_*}{r^2} = \frac{4\pi^2\, r}{P^2}$$
  \vskip8pt
  \textbf{Key insight:} The planet's mass cancels --- the orbit depends only on the central mass and the orbital radius.
\end{frame}

% ── Derivation step 2 ──────────────────────────────────────
\begin{frame}{Step 2: Solve for Stellar Mass}
  Rearranging:
  $$\boxed{M_* = \frac{4\pi^2\, r^3}{G\, P^2}}$$
  \vskip10pt
  This is \textbf{Newton's form of Kepler's third law} (for $M_p \ll M_*$).
  \vskip10pt
  \begin{exampleblock}{Note}
    For elliptical orbits, replace $r$ with the semi-major axis $a$.\\
    The general case requires the vis-viva equation (Lecture~2).
  \end{exampleblock}
\end{frame}

% ── Application: numbers ────────────────────────────────────
\begin{frame}{Application: Earth's Orbital Parameters}
  \centering
  \begin{tabular}{l l}
    \toprule
    \textbf{Quantity} & \textbf{Value} \\
    \midrule
    Semi-major axis $a_\oplus$ & $1.496 \times 10^{11}$~m \\
    Orbital period $P_\oplus$  & $3.156 \times 10^{7}$~s \\
    Gravitational constant $G$ & $6.674 \times 10^{-11}$~m$^3$\,kg$^{-1}$\,s$^{-2}$ \\
    \bottomrule
  \end{tabular}
\end{frame}

% ── Application: result ─────────────────────────────────────
\begin{frame}{Result: Solar Mass}
  $$\Msun = \frac{4\pi^2 \times (1.496 \times 10^{11})^3}{6.674 \times 10^{-11} \times (3.156 \times 10^{7})^2}$$
  \vskip10pt
  \keyresult{$$\Msun \approx 1.99 \times 10^{30} \text{~kg}$$}
  \vskip10pt
  Accepted value: $\Msun = 1.989 \times 10^{30}$~kg
  \vskip6pt
  A remarkably accurate estimate from just \textbf{two measurable quantities}: $a$ and $P$.
\end{frame}

% ── Planet-to-star mass ratio ───────────────────────────────
\begin{frame}{Planet-to-Star Mass Ratio}
  The precise form of Kepler's third law gives $M_* + M_p$, not $M_*$ alone.
  \vskip4pt
  The approximation $M_* + M_p \approx M_*$ works because:
  \vskip8pt
  \begin{itemize}
    \item Jupiter: $\Mjup \approx 318\;\Mearth \approx 1.90 \times 10^{27}$~kg
    $$\frac{\Mjup}{\Msun} \approx 9.5 \times 10^{-4}$$
    \item Total mass of all 8 planets: $\approx 446\;\Mearth$
    $$\frac{M_\mathrm{planets}}{\Msun} \approx 1.3 \times 10^{-3}$$
  \end{itemize}
\end{frame}

% ── Mass concentration ──────────────────────────────────────
\begin{frame}{Mass Budget of the Solar System}
  \keyresult of the solar system's total mass.\\[6pt]
    Jupiter alone accounts for \textbf{71\%} of the planetary mass.%
  }
  \vskip14pt
  \begin{itemize}
    \item This extreme concentration of mass in the central star is a \textbf{fundamental property} of planetary systems
    \item Planet formation theory must explain this (Lecture~2)
  \end{itemize}
\end{frame}


% ════════════════════════════════════════════════════════════
\section{Comparative Planetology}
% ════════════════════════════════════════════════════════════

% ── Methodology ─────────────────────────────────────────────
\begin{frame}{Comparative Planetology as a Methodology}
  \begin{itemize}
    \item We cannot perform controlled experiments on entire worlds
    \item \textbf{Comparative planetology:} treat planets as a natural set of experiments
    \item Similar objects subjected to different conditions
    \item By comparing cases, we isolate which differences arise from:
    \begin{itemize}
      \item Distance to the Sun
      \item Planetary mass
      \item Internal activity
      \item Historical contingency
    \end{itemize}
  \end{itemize}
\end{frame}

% ── Venus/Earth/Mars comparison ─────────────────────────────
\begin{frame}{Venus, Earth, Mars: Three Divergent Siblings}
  \centering
  \small
  \begin{tabular}{l c c c}
    \toprule
    \textbf{Property} & \textbf{Venus} & \textbf{Earth} & \textbf{Mars} \\
    \midrule
    Surface temperature & 460\,\textdegree C & 15\,\textdegree C & $-60$\,\textdegree C \\
    Surface pressure & 90~bar & 1~bar & 0.006~bar \\
    Dominant atmosphere & $\mathrm{CO_2}$ (96.5\%) & $\mathrm{N_2}$/$\mathrm{O_2}$ (99\%) & $\mathrm{CO_2}$ (95\%) \\
    Magnetic field & None & Strong dipole & Remnant crustal \\
    Tectonics & Episodic resurf. & Plate tectonics & Stagnant lid \\
    \bottomrule
  \end{tabular}
  \vskip10pt
  Formed in the same disk, similar compositions --- \textbf{why did they diverge?}
\end{frame}

% ── Moons as comparison sets ────────────────────────────────
\begin{frame}{Moons as Natural Experiments}
  \begin{itemize}
    \item \textbf{Galilean moons:} Io, Europa, Ganymede, Callisto\\
          All orbit Jupiter, but differ in composition and tidal heating
    \item A single variable (distance from Jupiter $\rightarrow$ tidal heating) drives vastly different geological outcomes:
    \begin{itemize}
      \item Io: most volcanically active body in the solar system
      \item Europa: subsurface liquid water ocean
      \item Ganymede: internal ocean + own magnetic field
      \item Callisto: ancient, undifferentiated surface
    \end{itemize}
  \end{itemize}
  \vskip8pt
  This comparative approach now extends to \textbf{thousands of exoplanets} (Lecture~13).
\end{frame}


% ════════════════════════════════════════════════════════════
\section{Observational Techniques}
% ════════════════════════════════════════════════════════════

% ── Telescopes ──────────────────────────────────────────────
\begin{frame}{Telescopes}
  \textbf{Ground-based:}
  \begin{itemize}
    \item Optical imaging: surface and atmospheric features
    \item Infrared spectroscopy: thermal emission, atmospheric composition
    \item Radio: subsurface properties, atmospheric dynamics
    \item Adaptive optics (VLT, Keck): near-space-telescope resolution
  \end{itemize}
  \vskip8pt
  \textbf{Space-based:}
  \begin{itemize}
    \item Hubble: atmospheric monitoring of giant planets for decades
    \item JWST (L2 point): infrared $\rightarrow$ primary tool for exoplanet atmospheres
  \end{itemize}
\end{frame}

% ── Spacecraft types ────────────────────────────────────────
\begin{frame}{Spacecraft Exploration}
  In order of increasing complexity and cost:
  \vskip6pt
  \begin{description}
    \item[Flyby] Brief encounter; e.g., Voyager~2 at Neptune (1989)
    \item[Orbiter] Long-term monitoring; e.g., Cassini at Saturn (2004--2017)
    \item[Lander] In situ measurements; e.g., Viking~1 on Mars (1976)
    \item[Rover] Mobile surface exploration; e.g., Perseverance on Mars (2021--)
    \item[Sample return] Material returned to Earth labs; e.g., Hayabusa2 from Ryugu (2020)
  \end{description}
\end{frame}

% ── Remote sensing methods ──────────────────────────────────
\begin{frame}{Remote Sensing Methods}
  \begin{description}
    \item[Spectroscopy] Identifies materials by absorption/emission features\\
    $\rightarrow$ atmospheric composition, surface mineralogy, thermal properties

    \vskip4pt
    \item[Radar] Penetrates clouds, maps surface topography\\
    $\rightarrow$ Magellan at Venus, MARSIS on Mars (subsurface features)

    \vskip4pt
    \item[Gravimetry] Maps gravity field from orbital perturbations\\
    $\rightarrow$ interior density variations, subsurface mass concentrations

    \vskip4pt
    \item[Magnetometry] Measures magnetic fields\\
    $\rightarrow$ dynamo activity, interior conductivity
  \end{description}
\end{frame}

% ── Key missions table (part 1) ─────────────────────────────
\begin{frame}{Landmark Missions (I)}
  \centering
  \small
  \begin{tabular}{l l l l}
    \toprule
    \textbf{Mission} & \textbf{Year} & \textbf{Target} & \textbf{Key achievement} \\
    \midrule
    Mariner 2       & 1962 & Venus  & First planetary flyby \\
    Mariner 4       & 1965 & Mars   & First close-up images \\
    Apollo 11       & 1969 & Moon   & First crewed landing \\
    Venera 7        & 1970 & Venus  & First surface landing \\
    Viking 1 \& 2   & 1976 & Mars   & First Mars landers \\
    Voyager 1 \& 2  & 1977 & Outer  & Grand tour \\
    Galileo         & 1995 & Jupiter & Europa's ocean \\
    \bottomrule
  \end{tabular}
\end{frame}

% ── Key missions table (part 2) ─────────────────────────────
\begin{frame}{Landmark Missions (II)}
  \centering
  \small
  \begin{tabular}{l l l l}
    \toprule
    \textbf{Mission} & \textbf{Year} & \textbf{Target} & \textbf{Key achievement} \\
    \midrule
    Cassini-Huygens & 2004 & Saturn   & Titan landing, Enceladus plumes \\
    Spirit/Opportunity & 2004 & Mars  & Evidence for past water \\
    New Horizons    & 2015 & Pluto    & First Pluto flyby \\
    Hayabusa2       & 2020 & Ryugu    & Asteroid sample return \\
    Perseverance     & 2021 & Mars    & Sample caching \\
    JWST            & 2022 & Exoplanets & Atmospheric characterisation \\
    Europa Clipper  & 2024 & Europa   & Ocean habitability \\
    \bottomrule
  \end{tabular}
\end{frame}


% ════════════════════════════════════════════════════════════
\section{Recent Advances \& Outlook}
% ════════════════════════════════════════════════════════════

% ── JWST ────────────────────────────────────────────────────
\begin{frame}{James Webb Space Telescope}
  \begin{itemize}
    \item Launched December 2021; operating at Sun--Earth L2
    \item First characterisation of atmospheres on \textbf{rocky exoplanets}
    \item Thermal emission measurements of TRAPPIST-1 planets\\
          (Greene et al.\ 2023)
    \item First direct constraints on whether Earth-sized planets around other stars retain atmospheres
    \item Infrared spectroscopy: transmission \& emission
  \end{itemize}
\end{frame}

% ── Recent missions ─────────────────────────────────────────
\begin{frame}{Recent \& Upcoming Missions}
  \begin{itemize}
    \item \textbf{OSIRIS-REx} (2023): Returned samples from asteroid Bennu\\
          $\rightarrow$ hydrated minerals, organic compounds

    \item \textbf{Europa Clipper} (2024): En route to Jupiter's moon Europa\\
          $\rightarrow$ investigate subsurface ocean \& habitability

    \item \textbf{JUICE} (2023$\rightarrow$2031): ESA mission to Jupiter system\\
          $\rightarrow$ Ganymede, Europa, Callisto

    \item \textbf{Perseverance}: Caching samples in Jezero crater

    \item \textbf{Dragonfly} ($\sim$2028): Nuclear-powered drone on Titan\\
          $\rightarrow$ surface chemistry exploration
  \end{itemize}
\end{frame}

% ── Course roadmap ──────────────────────────────────────────
\begin{frame}{What's Coming: Course Roadmap}
  \small
  \begin{columns}[T]
    \column{0.48\textwidth}
    \textbf{Weeks 1--4 (Foundations):}
    \begin{itemize}
      \item[L2] Formation \& orbits
      \item[L3] Heat \& energy transport
      \item[L4] Differentiation \& magnetospheres
      \item[L5] Atmospheres I
      \item[L6] Atmospheres II
      \item[L7] Surfaces
      \item[] \textit{Mid-term exam}
    \end{itemize}

    \column{0.48\textwidth}
    \textbf{Weeks 5--7 (Applications):}
    \begin{itemize}
      \item[L8] Interiors
      \item[L9] Earth \& Venus
      \item[L10] Mercury \& Mars
      \item[L11] Gas \& ice giants
      \item[L12] Small bodies
      \item[L13] Exoplanets
      \item[L14] Synthesis \& astrobiology
    \end{itemize}
  \end{columns}
\end{frame}

% ── Summary ─────────────────────────────────────────────────
\begin{frame}{Summary}
  \begin{enumerate}
    \item Planetary science asks three driving questions:\\
          formation, habitability, life beyond Earth
    \item The field evolved from ancient stargazing $\rightarrow$ telescopes $\rightarrow$ space missions $\rightarrow$ exoplanet surveys
    \item The solar system has eight planets spanning a huge range in mass, size, and composition
    \item From just orbital period and distance, we can measure the Sun's mass: $\Msun \approx 2 \times 10^{30}$~kg
    \item Comparative planetology --- treating worlds as natural experiments --- is our most powerful tool
    \item We are living in a golden age of planetary exploration
  \end{enumerate}
\end{frame}

% ── Final slide ─────────────────────────────────────────────
\begin{frame}[plain,noframenumbering]
  \begin{tikzpicture}[remember picture, overlay]
    \fill[ipsPrimary] (current page.north west) rectangle (current page.south east);
  \end{tikzpicture}
  \vfill
  \begin{center}
    {\color{ipsAccent}\Large\bfseries Questions?}
    \vskip16pt
    {\color{ipsLightFill}\normalsize Lecture notes: \texttt{formingworlds.github.io/IntroductionToPlanetaryScience}}
  \end{center}
  \vfill
\end{frame}

\end{document}
